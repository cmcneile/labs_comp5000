\section*{Question 1: python revision}

An engineer, who is designing optical fibers, asks you to solve
Snell's law for $\theta_{air}$ given values of $\theta_{glass}$ (input
from the user)),
$n_{air}$ = 1 (fixed), and $n_{glass}$ = 1.4 (fixed).

$$
n_{air} \sin( \theta_{air} ) = n_{glass} \sin( \theta_{glass} ) 
$$

The engineer wants to be able to input a value for
$\theta_{glass}$ in degrees and ouput a value
of $\theta_{air}$ in degress. For some values of $\theta_{glass}$
there is no solution for $\theta_{air}$ so your script should
deal with this situation (I suggest you use the \textbf{if} statement
to do this).
The engineer gives
you the script \textbf{Q1\_lab1\_hint.py} as a starting
point.

To develop the script I suggest you follow an idea called
test driven development
(\url{https://en.wikipedia.org/wiki/Test-driven_development}).
You design simple test cases by hand and calculator and
then use the test cases to build up the script. In the above
example, you can have test cases to check the computed value of
$\theta_{air}$, but you can also check intermediate steps, such as
converting degrees to radians.


I suggest the following steps to modify the code.

\begin{itemize}

\item Download and run the python script \textbf{Q1\_lab1\_hint.py} from
  the DLE.

\item Develop a test case using a calculator for $\theta_{glass}$ =
  30\si{\degree}, by solving the above equation for $\theta_{air}$.

\item Develop a test case where there is no solution.

\item Modify the script so that it does what the engineer wants.

\item Test the script using the two test cases.

\end{itemize}


% LocalWords:  ouput degress py
